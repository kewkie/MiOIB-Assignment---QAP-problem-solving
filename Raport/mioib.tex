\documentclass[12pt,polish,msc]{article}

% \usepackage{polski} %moze wymagac dokonfigurowania latexa, ale jest lepszy ni� standardowy babel'owy [polish] 

\usepackage{babel}
\usepackage[cp1250]{inputenc} 
\usepackage[OT4]{fontenc} 
\usepackage{graphicx,color} %include pdf's (and png's for raster graphics... avoid raster graphics!) 
\usepackage{url} 

%% Zmiana rozmiar�w strony tekstu
\addtolength{\voffset}{-1cm}
\addtolength{\hoffset}{-1cm}
\addtolength{\textwidth}{2cm}
\addtolength{\textheight}{2cm}

\begin
%bardziej zyciowe parametry sterujace rozmieszczeniem rysunkow
\renewcommand{\topfraction}{.85}
\renewcommand{\bottomfraction}{.7}
\renewcommand{\textfraction}{.15}
\renewcommand{\floatpagefraction}{.66}
\renewcommand{\dbltopfraction}{.66}
\renewcommand{\dblfloatpagefraction}{.66}
\setcounter{topnumber}{9}
\setcounter{bottomnumber}{9}
\setcounter{totalnumber}{20}
\setcounter{dbltopnumber}{9}
\end

% w�asny bullet list z malymi odstepami
% \newenvironment{tightlist}{
% \begin{itemize}
%  \setlength{\itemsep}{1pt}
%  \setlength{\parskip}{0pt}
%  \setlength{\parsep}{0pt}}
%{\end{itemize}}




\graphicspath{{../plotstex/}}

%\title{Sprawozdanie z laboratorium:\\Metaheurystyki i Obliczenia Inspirowane Biologicznie}
%\author{}
%\date{}


\begin{document}

\thispagestyle{empty} %bez numeru strony

\begin{center}
{\large{Sprawozdanie z laboratorium:\\
Metaheurystyki i Obliczenia Inspirowane Biologicznie}}

\vspace{3ex}

Cz�� I: Algorytmy optymalizacji lokalnej, problem ATSP
%Cz�� II: Algorytmy optymalizacji lokalnej i globalnej, problem QAP
%Cz�� III: Eksperyment: ... (prezentacj� mo�na zrobi� w LaTeX/klasa beamer)

\vspace{3ex}
{\footnotesize\today}

\end{center}


\vspace{10ex}

Prowadz�cy: dr in�. Maciej Komosi�ski

\vspace{5ex}

Autorzy:
\begin{tabular}{lllr}
\textbf{Sebastian Kochman} & inf75898 & SKiSR & sebastian.kochman@gmail.com \\
\textbf{Maciej Kokoci�ski} & inf75899 & SKiSR & maciej.kokocinski@gmail.com \\
\end{tabular}

\vspace{5ex}

Zaj�cia poniedzia�kowe, 11:45.


\newpage





\section{Wst�p}


\section{Algorytmy}


\subsection{Random}


\subsection{Nearest neighbor}


\subsection{Local search}


\subsubsection{Steepest}

\subsubsection{Greedy}
\label{sec-greedy}

\subsection{Operatory s�siedztwa}
\label{sec-operatory}

\subsubsection{Swap}

\subsubsection{Invert}

\subsubsection{Permute}

\section{Eksperymenty}
\label{sec-eksperymenty}

\subsection{Odleg�o�� od optimum}

\label{odleglosc_od_optimum}

\subsection{Czas dzia�ania}

\clearpage

\subsection{Efektywno��}

\subsection{Liczba iteracji}


\subsection{Liczba przejrzanych rozwi�za�}


\subsection{Zale�no�� jako�ci rozwi�zania ko�cowego od jako�ci rozwi�zania pocz�tkowego}


\subsection{Wielokrotne uruchamianie}


\subsection{Podobie�stwo znajdowanych rozwi�za�}

\subsection{Badanie wp�ywu losowo�ci}
\label{sec-randomness}

\section{Wnioski}

\section{Propozycje ulepsze�}

%%%%%%%%%%%%%%%% literatura %%%%%%%%%%%%%%%%

%\bibliography{mioib}
%\bibliographystyle{plain}


\end{document}

